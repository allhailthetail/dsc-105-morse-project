\documentclass[11pt]{article}

%% Page Dimensions %%
\usepackage[letterpaper, margin=1in]{geometry} %% Letter page setup

%%%%%%%%%%%%%%%%%%%%%%%%%%%%%%%%%%%
%% Packages %%
\usepackage{amsmath} % Necessary
\usepackage{amssymb} % Necessary
\usepackage{amsthm} % Necessary
\usepackage{amsfonts} % Necessary
\usepackage{hyperref} % Automatically turns all internal references to links
\usepackage{fancyhdr} % Allows for headers and footers

\newcounter{qcounter}
\setcounter{qcounter}{1}
\setlength{\parindent}{0pt}

\newcommand{\q}{\newpage\section*{Question {\theqcounter}} \stepcounter{qcounter}}

%%%%%%%%%%%%%%%%%%%%%%%%%%%%%%%%%%%

\pagestyle{fancy}
\chead{Understanding the PARIS Standard in CW \\ Matthew Younger \today}

\begin{document}
	\section*{Calculating the time unit $\tau$:}
	
	CW relies on a tightly-controlled spacing of $1$'s and $0$'s in order to convey a message accurately. Fudging this would mean an unintelligible message. "Did I code \texttt{I (**)} or \texttt{E E (* *)}? As such, folks decided that the the spacing of all distinct morse elements is as follows:
	
	\vspace{10pt}
	
		% Table of elements and units in terms of Tau:
		\begin{tabular}{|c|l|}
			\hline 
			 element & units \\
			\hline
			dot & $\tau$ \\
			dash & $3 \tau$ \\
			\textless e-spc\textgreater & $\tau$\\
			\textless c-spc\textgreater & $3 \tau$ \\
			\textless w-spc\textgreater & $7 \tau$ \\
			\hline
		\end{tabular}
		
	\vspace{10pt}
	
	% Now, show how we arrive at 6/5 * 1/WPM = Tau...
	% Include a graph of Tau vs WPM to show asymtope... 
	Now, to why it's called the PARIS standard: \\	
	The word PARIS has $43\tau + 7\tau = 50\tau$ elements in it. Convince yourself of this, which isn't terribly hard.
	The implication is that the \textit{speed} that you send this word at, your $n$ \textbf{WPM}, relates to $\tau$:
	
	$ n_{wpm} = \frac{\textit{total time}}{\textit{time per word}} \textit{words}
	= \frac{60sec}{T_{word}} (1_{word})
	= \frac{60}{50\tau} $ \\
	$ \therefore \hspace*{10pt} 
	\tau = \frac{6}{5n}$
	
\end{document}
